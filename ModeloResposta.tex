\documentclass[a4paper,12pt]{article}
\usepackage{graphicx}
\usepackage[a4paper, total={6in, 9in}]{geometry}
\usepackage[T1]{fontenc}
\usepackage[utf8]{inputenc}
\usepackage{graphicx}
\usepackage{tikz}
\usepackage{float}
\usepackage{mathtools}
\usepackage{fancyhdr}
\usepackage{caption}
\usepackage{textgreek}
\usepackage{yfonts}
\usepackage{amssymb}
\usepackage{hyperref}
\usepackage{amsmath}
\usepackage{systeme}
\usepackage{minted}
\hypersetup{
    colorlinks=true,
    linkcolor=blue,
    filecolor=magenta,
    urlcolor=cyan,
    pdftitle={Overleaf Example},
    pdfpagemode=FullScreen,
    }

\urlstyle{same}
\let\empty\varnothing
\let\eqv\Longleftrightarrow
\usepackage{longtable}
\graphicspath{./images}
\pagestyle{fancy}
\date{Março 2022}
\title{ \\ \large {Initial Report}}
\author{João Barreiros C. Rodrigues}

\begin{document}
        \pagenumbering{gobble}
        \begin{titlepage} % Suppresses displaying the page number on the title page and the subsequent page counts as page 1
        \newcommand{\HRule}{\rule{\linewidth}{0.5mm}} % Defines a new command for horizontal lines, change thickness here
        \center % Centre everything on the page
        \textsc{\LARGE Instituto Superior Técnico}\\[1.5cm] % Main heading such as the name of your university/college
        \textsc{\Large Probabilidades e Estatística}\\[0.25cm]
        \HRule\\[0.4cm]
        {\LARGE\bfseries Projecto Computacional}\\[0.4cm] % Title of your document
        {\huge\bfseries Pergunta}\\[0.4cm] % Title of your document
        \HRule\\[1.5cm]\
        João \textsc{Barreiros C. Rodrigues},nº 99968 , LEEC-IST\\
        \vfill\vfill\vfill % Position the date 3/4 down the remaining page
        {\large 4º trimestre 2022} % Date, change the \today to a set date if you want to be precise
        \vfill % Push the date up 1/4 of the remaining page
\end{titlepage}
        \pagenumbering{arabic}
        \newpage
        \tableofcontents
        \clearpage
	\section{Código}
	\inputminted{r}{Module.R}
	\section{Outputs}
	\section{Comentários Adicionais}

\end{document}







